\documentclass{article}
\setlength{\parindent}{0em} 
\setlength{\parskip}{1.4ex} 
\usepackage[danish]{babel} 
\usepackage[utf8]{inputenc}
\usepackage{amsfonts}

\usepackage{fancyhdr}

\pagestyle{fancy}
\fancyhf{}
\rhead{Syddansk Universitet}
\lhead{Reeksamen februar 2015}
\rfoot{ \thepage}

\title{Reeksamen februar 2015}
\author{Vivi Nguyen Chi Tran}
\date{15. november 2020}

\begin{document}
\maketitle
\newpage
\section*{Opgave 1}
I det følgende laver vi $U = \{1,2,3,...,15\}$ være universet (universal set).\\
Betragt de to mængder
\begin{center}
$A = \{2n | n \in S\}$ og $B = \{3n + 2 | n \in S\}  $
\end{center}
hvor $S = \{1,2,3,4\}.$\\
\\
\\
a) $A = \{2,4,6,8\}$\\
Alle lige tal udvundet af S tilhørende universet.
\\
\\
b) $B = \{5,8,11,14\}$\\
Alle tal tilhørende universet ganget med 3 med 2 lagt til.
\\
\\
c) $A \cap B = \{8\}$\\
Fællesmængden af A og B.
\\
\\
d) $A \cup B = \{2,4,5,6,8,11,14\}$\\
Foreningsmængden af A og B.
\\
\\
e) $A - B = \{2,5,6\}$\\
A minus alle de tal A har tilfælles med B.
\\
\\
f) $\bar{A} = \{1,3,5,7,9,10,11,12,13,14,15\}$ \\
Komplimentet til A.

\section*{Opgave 2}
a)
\\
1. Udsagnet er sandt. Der er et y for hver værdi af x der er større end x.
\\
2. Udsagnet er falskt. Der er eksistere mere end et og kun et y der er størrere end x.
\\
3. Udsagnet er falskt. Der findes ikke en værdi af y der er større end alle værdier af x.
\\
\\
b) Udtrykket negeres vha. De Morgans love og ser ud på følgende vis: \\
$$\exists x \in \mathbb{N}: \forall y \in \mathbb{N}: x \geq y
$$
\end{document}